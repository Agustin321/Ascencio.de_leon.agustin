\documentclass[14pt,letterpaper]{article}
\usepackage[utf8]{inputenc}
\title{EV 2.1 DISEÑO DE PUENTE H}
\author{Ascencio De Leon Agustin}
\usepackage[spanish]{babel}
\usepackage{graphicx}
\graphicspath{{img/}}
\usepackage[left=2.5cm,top=2.5cm,bottom=3cm,right=2.5cm]{geometry}


\begin{document}
\maketitle
\begin{figure}[h!]
\centering
\includegraphics[scale=.9]{upzmg}
\end{figure}
\newpage
\section{objetivo de la practica}
en esta practica se presenta la situacion en la que debemos diceñar un circuito en donde un boton haga girar un motor a un lado y otro boton lo haga girar a el lado opuesto.\\
\section{procedimiento}
1. lo promero que hicimos fue buscar un ejemplo en el cual basarnos para poder comenzar.\\
2. lo siguiente fue buscar los componentes y sus datos en el datasheet, para poder hacer los respectivos calculos.\\
3. despues de tener los calculos se procedio a armar el circuito y se hicieron pruebas.
\section{conclucion}
teniendo en cuenta lo aprendido, se reafirmaron datos ya vistos.\\
sin embargo se aclararon dudas como el porque gira a distintos sentidos dependiendo de la funcion de el circuito o de el puente H.\\
haciendo mas sencillo su diseño y armado.\\ 
\end{document}