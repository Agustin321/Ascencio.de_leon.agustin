\documentclass[11pt,a4paper]{article}
\usepackage{graphicx}
\usepackage{amsmath}
\usepackage{amssymb}
\usepackage{mathrsfs}
\usepackage{cancel}

\begin{document}
\begin{center}
\textbf{REPORTE 1 DE AVANCE PROYECTO}\\
.\\
DRON INTELIGENTE
\end{center}

\begin{figure}[h]
\centering
\includegraphics[width=12cm]{../PENIDNTE/upzmg.png} 
\end{figure}

\begin{center}
Maria de Lourdes Gomez Islas\\
Asencio de leon Agustín\\
Cruz Ramírez Jesús Osmar\\
González González eldrich johel\\
Calderón Hernández Richard\\
Partida López Ernesto Alonso\\
.\\
08-OCT-2019\\
Universidad Politecnica de La Zona Metropolitana de Guadalajara
\end{center}

\newpage 

\part{Planteamiento del problema}
El problema que se desea resolver con este proyecto es la inseguridad en los fraccionamientos mas vulnerables ante el crimen organizado y criminales de bajo nivel.
Los fraccionamientos como Chulavista o Santafé son muy vulnerables ante los crimenes, tanto así que ya se consideran de los fraccionamientos más peligrosos de Tlajomulco de Zuniga por lo cual se implementara un sistema de vigilancia de drones autónomos que monitorearan los policías en los  sectores en los que existe  falta de videovigilancia.
También se puede decir que el problema en el que ayuda sería hacia los fotógrafos ya que con este tipo de drones se pueden plasmar mejor las imágenes.

\section{Formular el problema}
Si el dron está volando en el aire y ve a alguien en peligro que pasaría?
La persona que en teoría esta monitoreando el dron o ese dron sería alertada del suceso y enviaria a la policía o a alguien que pueda resolver ese problema

¿El dron lo pueden usar otras personas?
Si, cualquier persona puede usar el dron desde un niño hasta un fotógrafo o para su uso específico que es la seguridad

\part{Objetivo del general}
Nuestro objetivo es elaborar un dron inteligente en la cual usaremos una raspberry pi zero W y un pequeño modulo de picamera para que este pueda seguir una una pelota de bolsillo o incluso un rostro que servirá de guía para ser controlado durante el vuelo, tambien es de facilitar el trabajo a los elementos de protección civil en rescates y recuperaciones de personas extraviadas en lugares de difícil acceso, mediante el GPS y un la ente que usaremos para captar en momento real lo que sucede en su trayecto y reconocimiento de esto mencionado anteriormente. Para este debemos conocer primeramente los materiales a utilizar como como los sensores y planeaciones de vuelo e estabilidad para que este tenga su funcionamiento correcto, para esto ahí que plantear las leyes de física que debemos repasar para provocar este efecto de estabilidad que buscamos para nuestro proyecto.  

\paragraph{Objetivo del proyecto:}


Este es el diseño de un dron totalemnte automatico e inteligente, los beneficios de este dron es, que ya no tendrá los pequeños accidentes que suelen ocurrir durante el proceso de vuelo los mas comunes son de golpear objetos tales como arboles y paredes, lo que planeamos a futuro con este proyecto es de darle una utilidad con el equipo de servicio de protección civil del estado de Guadalajara en proporcionarle un equipo para que este pueda monitorear y rastrear a personas extraviadas dentro de un parque ecológico, bosques, selvas y barrancos, tambien para hacer más rápida la localización de cuerpos extraviados dentro de lagunas, ríos y mar adentro, con el fin de asi facilitar el trabajo de estos elementos de protección civil.

\part{Justificacion}

El proyecto se penso con el fin de apoyar de apoyar a los elementos de proteccion de civil de Guadalajara, dentro de las labores de búsqueda y rescate de personas y de facilitar el recorrido de estos dentro de zonas de difícil acceso, por otra partes mas que nada para que se facilite el uso de control de este dron automatizado ya que contara con sensores de aproximación con los arboles asi evitando que se dañe nuestro dron.\\
Lo elegimos por hecho de que es prácticamente algo complejo y automatizado, de lo que tenemos conocimiento es que el personal de protección civil de nuestro estado no cuenta con algun apartao que que pueda ayudarles con estas tareas de rescate y busqueda ya que aun siguen poniendo eb riesgo la vida de su personlal, en cuando el proyecto sea aprobado por lasautoridades de proteccion de civil y cuando se registre el primer caso es donde aplicaremos el funcionamiento de nuestro dron.

\paragraph{Delimitacion:}


\begin{itemize}
\item Medicion Aprox. 45cm de distancia de ala a ala.
\item Peso Max. 2kg.
\item Altura de vuelo hasta 10m.
\item Tiempo de vuelo 1 hora antes de recargarce.
\item Alejamiento de el dispositivo de control 500m Aprox.
\end{itemize} 

\section{Posibles materiales y costos}

\begin{tabular}{|c|c|}
\hline
\textbf{Materiales} & \textbf{Costos}\\ \hline
Chasis o Frame & 700-800 \\ \hline
Motorizacion & 500 \\ \hline
PDB y Controladora & 250-2150 \\ \hline
Batería & 600 \\ \hline
Hélices & 200 \\ \hline
Emisora & 1000 \\ \hline
Equipo & 1600 \\ \hline
\end{tabular}



\section{Roles}

\begin{tabular}{|c|c|}

\hline
\textbf{Integrante} & \textbf{Rol} \\ \hline
Maria de Lourdes  & Encargada del Diseño \\ \hline
Ernesto Alonso & Encargado del Diseño \\ \hline
Jesús Osmar & Encargado de Impresion y Material \\ \hline
Hernández Richard & Encargado del Ensamblado \\ \hline
Maria de Lourdes & Calculos \\ \hline
Eldrich Johel & Parametros y Pruebas \\ \hline
Asencio de leon & Programacion \\ \hline
\end{tabular}

\subsection{Tiempos y Actividades}

\begin{figure}[h]
\centering
\includegraphics[width=13.5cm]{../DIAGRAMADEGANTT2.png} 
\caption{Diagrama de GANTT}
\end{figure}


\newpage

\part{Explicacion de Aportacion de las materias}

\begin{tabular}{|p{5.5cm}|p{7cm}|}
\hline 
\textbf{Materias de 4to} & \textbf{Aportacion al proyecto} \\ \hline
\textbf{ingles 4} & Es indispensable saber el lenguaje de  ingles ya que es el lengueje universal para programar \\ \hline
\textbf{etica profesional} & Si hablamos de Etica, el valor de proteger al progimo a la familia y a la humanidad  se emplea en este proyecto ya que se lleva a cabo un sistema de vigilancia. \\ \hline
\textbf{estructura y propiedades} & Nos aporto la informacion indicada para el tipo de material a utilizar para la resistencia  del modelo \\ \hline
\textbf{programacion de perifericos} & Utilizar los aprendizajes de la materia para ser aplicador en la programacion rotatorio hacia los motores usando una Raspberry pi zero W \\ \hline
\textbf{S. E. de interfaz} & Hacer uso de un modelo trifasico para los motores conectado al programa y calculando la division requerida de tension.  \\ \hline
\textbf{PLC} & Este proyecto se llevara a cavo un dron controlado, con indicaciones seguido de sensores para hacer sus funciones. \\ \hline
\end{tabular}

\paragraph{Desarrollo del proyecto:}
Reunir los requisitos para empezar el proyecto, proponiendo ideas para concluirlo.

\newpage

\cite{ruiperez2016diseno,}
\bibliographystyle{apalike}
\bibliography{ReporteBib}

\end{document}