\documentclass[14pt,letterpaper]{article}
\usepackage[utf8]{inputenc}
\title{proyecto anual "compberry"}
\author{equipo helios}
\usepackage[spanish]{babel}
\usepackage[left=5cm,top=2.5cm,bottom=3cm,right=5cm]{geometry}

\begin{document}
\maketitle
Ascencio De León Agustín.
\linebreak
Cruz Ramírez Jesús Osmar.
\linebreak
González González Eldrich Johel.
\linebreak
Partida López Ernesto Alonso.
\linebreak
Calderón Hernández Richard
\linebreak
\linebreak
\linebreak
Universidad Politécnica De La Zona Metropolitana De Guadalajara
20 de septiembre del 2019
\newpage
\section{ Planteamiento del problema} 
Durante estos últimos meses, el equipo a notado que un pequeño grupo de alumnos de todas las carreras no cuentan con una computadora portátil, lo cual afecta en su conocimiento y aprendizaje; y a su vez entorpece y evita que se desempeñen correctamente, por lo cual se busca una manera de que este grupo de alumnos logren obtener una computadora para su uso personal a un bajo costo y con un desempeño adecuado para lograr realizar las tareas y/o trabajos  dejados en clase.
\linebreak
\linebreak
\linebreak
\section{específico} 
Mediante la culminación del proyecto, se buscará beneficiar a aquellos alumnos que no cuentan con los recursos suficientes para adquirir una computadora debido a los altos costos que existen en el mercado actual. No será una computadora de alto nivel, pero si será lo suficiente para que el alumno desempeñe sus actividades escolares.
\linebreak
\linebreak
\linebreak
\section{Objetivo general del proyecto} 
Elaborar una computadora de escritorio, la cual después de un riguroso estudio se convierta en una computadora portátil, para que el alumno la pueda transportar sin ningún problema, para esto será necesario optar por el mejor material para la cubierta tanto de la computadora de escritorio como la laptop.
\linebreak
\linebreak
\linebreak
\section{Justificación}
Será una buena opción para aquellos alumnos, compañeros o personas que no cuentan con el recurso suficiente para adquirir una computadora portátil. Además, el proyecto en un futuro podría competir con computadoras de mediano poder e incluso de alto poder si se le da el seguimiento adecuado. Pero principalmente se ve por el veneficio de los alumnos que no cuentan con un equipo para realizar sus tareas o trabajos.

Se planea llegar hasta la creación de una laptop la cual se portátil, pero se comenzara con una computadora de escritorio y de ser posible, que la computadora logre competir con computadoras de empresas con alto renombre como HP, Toshiba, Asus, Alienware, Mac, Lenovo entre otras.
\linebreak
\linebreak
\linebreak
\section{Matriz de posibles materiales y costos} 
    • Un Raspberry Pi(preferiblemente el 3 ya que cuenta con Bluetooth y receptor WiFi integrado)=...........................................................1200.00
    \linebreak
    •protector para Raspberry Pi=....................................110 
    \linebreak
    •Disipadores(recomendamos 2)=.......................................165
    \linebreak
    •Tarjeta de memoria microSD(sera el disco duro de nuestra mini PC)=...500
    \linebreak
    •Fuente de alimentación=.........................................250
    \linebreak
    •Monitor=.......................................................1000
    \linebreak
    •Teclado=.....................................................150
    \linebreak
    •Mouse=......................................................100
    \linebreak
    •Total=...................................................3,475-3,500
\linebreak
\linebreak
\linebreak
\section{Matriz de roles} 
Agustín:  la instalación y correcto funcionamiento del sistema operativo
Osmar: Cotización de precios de los productos necesarios, así como la obtención de los productos (proveedor) 
Eldrich: ayudar en el armado, y obtención de los productos
Richard: Armar la computadora con todos los materiales
Alonso: realizar los reportes correspondientes, armado, cotización e instalación del software.
\linebreak
\linebreak
\linebreak
\section{Desarrollo del proyecto}
    1. Se comenzó con la cotización de los materiales, lo cual fue personalmente y directamente con las tiendas.
    \linebreak
    2. Se adquirió la tarjeta raspberry pi 3b+, junto con su protector.
    \linebreak
    3. Se comienza buscando el sistema operativo que sea mejor para la futura  computadoras.
    \linebreak
    4. Se reparten los roles de que hará cada integrante.
    \linebreak
    5. ….
    \linebreak
    6. ….
    \linebreak
    7. ….
    \linebreak
    8. ….
    \linebreak
    9. ….
    
\end{document}